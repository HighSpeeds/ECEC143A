\UseRawInputEncoding
\documentclass[12pt]{article}
\title{ECE C143A Homework 6}
\usepackage{subcaption}
\author{Lawrence Liu}
\usepackage{graphicx}
\usepackage{amsmath}

\newcommand{\Laplace}{\mathscr{L}}
\setlength{\parskip}{\baselineskip}%
\setlength{\parindent}{0pt}%
\usepackage{xcolor}
\usepackage{listings}
\definecolor{backcolour}{rgb}{0.95,0.95,0.92}
\usepackage{amssymb}
\lstdefinestyle{mystyle}{
    backgroundcolor=\color{backcolour}}
\lstset{style=mystyle}

\begin{document}
\maketitle
\section*{Problem 1}
\subsection*{(a)}
False, the Na+ channel opens first
\subsection*{(b)}
False, only Na+ serve to depolarize the cell.
\subsection*{(c)}
True
\subsection*{(d)}
False, EEG's cannot record action potentials.
\subsection*{(e)}
False because $\lambda$ does not vary with time and a possion process is memoryless.
\subsection*{(f)}
False, if the Fano factor is greater than one, the firing variance is greater than the firing mean
\subsection*{(g)}
True
\subsection*{(h)}
False, the 
\subsection*{(i)}
False
\subsection*{(j)}

\subsection*{(m)}
False it is a low pass.
\subsection*{(n)}
False, good for visual bad for motor
\subsection*{(o)}
False, Absolute not relative


\section*{Problem 2}
\subsection*{(a)}
$f(\theta)$ reaces a max at $\theta=\theta_0$ therefore this
is the prefered direction.
\subsection*{(b)}
No because the values of the tuning curve would all be negative
\subsection*{(c)}
\begin{align*}
    \cos(\theta-\theta_0)&=e^{j(\theta-\theta_0)}\\
    &=(\cos(\theta)+j\sin(\theta))(\cos(\theta_0)-j\sin(\theta_0))\\
    &=\cos(theta)\cos(theta_0)+\sin(\theta)\sin(\theta_0)
\end{align*}
\subsection*{(d)}
$$k_0=c_0$$
$$k_1=c_1\sin(\theta_0)$$
$$k_2=c_1\cos(\theta_0)$$
\subsection*{(e)}
We have
$$y_0=25=k_0+k_2$$
$$y_{120}=70=k_0+\frac{k_1}{2}-\frac{k_2\sqrt{3}}{2}$$
$$y_{240}=10=k_0-\frac{k_1}{2}-\frac{k_2\sqrt{3}}{2}$$

Therefore we have
$$y_{120}+y_{240}=2k_0-k_2\sqrt{3}$$
$$2y_0-y_{120}-y_{240}=(2+\sqrt{3})k_2$$
$$k_2=\boxed{\frac{2y_0-y_{120}-y_{240}}{2+\sqrt{3}}}$$
$$k_0=\boxed{y_0-\frac{2y_0-y_{120}-y_{240}}{2+\sqrt{3}}}$$
$$k_1=\boxed{y_{240}-y_0+\frac{2y_0-y_{120}-y_{240}}{2+\sqrt{3}}+\frac{4y_0-2y_{120}-2y_{240}}{2\sqrt{3}+3}}$$
\subsection*{(f)}


\end{document}