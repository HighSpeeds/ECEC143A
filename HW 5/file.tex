\UseRawInputEncoding
\documentclass[12pt]{article}
\title{ECE C143A Homework 4}
\usepackage{subcaption}
\author{Lawrence Liu}
\usepackage{graphicx}
\usepackage{amsmath}
\usepackage{bm}
\usepackage{pdfpages}
\newcommand{\Laplace}{\mathscr{L}}
\setlength{\parskip}{\baselineskip}%
\setlength{\parindent}{0pt}%
\usepackage{xcolor}
\usepackage{listings}
\definecolor{backcolour}{rgb}{0.95,0.95,0.92}
\usepackage{amssymb}
\lstdefinestyle{mystyle}{
    backgroundcolor=\color{backcolour}}
\lstset{style=mystyle}

\begin{document}
\maketitle
\section*{Problem 1}
\subsection*{(a)}
$P(N=0)=1-0.25=\boxed{0.75}$
\subsection*{(b)}
We want $P(N=0|R=0)$, we know that
$$P(R=1|N=0)=P(E=0)P(R=1|E=0,N=0)+P(E=1)P(R=1|E=1,N=0)=0.01$$
$P(R=0|N=0)=0.99$ and thus from bayes law we have 
\begin{align*}
    P(N=0|R=0)&=P(R=0|N=0)\frac{P(N=0)}{P(R=0)}\\
    &=0.9\frac{0.1\cdot 0.75}{P(R=0)}
\end{align*}
To find $P(R=0)$ we must find $P(R=1)$,
\begin{multline*}
    P(R=1)=P(E=1)P(N=1)P(R=1|E=1,N=1)\\+P(E=1)P(N=0)P(R=1|E=1,N=0)
    \\+P(E=0)P(N=1)P(R=1|E=0,N=1)\\+P(E=0)P(N=0)P(R=1|E=0,N=0)
\end{multline*}
\begin{multline*}
    P(R=1)=0.9\cdot0.25\cdot 1+0.1\cdot0.75\cdot0.1+0.1\cdot 0.25\cdot 0.1=0.235
\end{multline*}
Therefore $P(R=0)=1-P(R=1)=0.765$, thus
$$P(N=0|R=0)=0.99\frac{0.75}{P(R=0)}=\boxed{0.97}$$
\subsection*{(c)}
We have
\begin{align*}
    P(E=0,N=0|R=0)&=\frac{P(E=0,N=0,R=0)} {P(R=0)}\\
    &=\frac{P(R=0|E=0,N=0)P(E=0)P(N=0)} {P(R=0)}\\
    &=\frac{(1-P(R=1|E=0,N=0))P(E=0)P(N=0)} {P(R=0)}\\
    &=\frac{0.9\cdot0.1\cdot0.75}{0.765}\\
    &=\boxed{0.088}
\end{align*}
This intuitively makes sense because we need two conidtions to occur, equipment broken and neuron spike not occuring.
\subsection*{(d)}
Let us consider the case $E=1$, $N=0$ given $R=1$ we have
$$P(E=1,N=0|R=1)=\frac{P(R=1|E=1,N=0)P(E=1)P(N=0)}{P(R=1)}$$
Since $P(R=1|E=1,N=0)=0$, we thus have $P(E=1,N=0|R=1)=0$. However 
\begin{align*}
    P(E=1|R=1)&=P(R=1|E=1)\frac{P(E=1)}{P(R=1)}\\
    &=(P(R=1|E=1,N=0)P(N=0)+\\
    &P(R=1|E=1,N=1)P(N=1))\frac{P(E=1)}{P(R=1)}
\end{align*}
This therefore $P(E=1|R=1)>1$, likewise
\begin{align*}
    P(N=0|R=1)&=P(R=1|N=0)\frac{P(N=0)}{P(R=1)}\\
    &=(P(R=1|E=1,N=0)P(E=1)+\\
    &P(R=1|E=0,N=0)P(E=0))\frac{P(N=0)}{P(R=1)}
\end{align*}
This therefore $P(E=1|R=1)>1$, therfore, $P(E=1|R=1)P(N=0|R=1)>0$ and is not equal to $P(E=1,N=0|R=1)$ therefore they are conditionally dependent.
Ie they are not independent given R
\section*{Problem 2}
\subsection*{(a)}
$$\boxed{P(a,b,c,d)=P(c)P(a|c)P(d)P(b|a,d)}$$
\subsection*{(b)}
\begin{align*}
    P(C,D)&=\sum_{A}P(C,D,A)\\
    &=\sum_{A}P(A)P(C|A)P(D|A)\\
    &=\sum_{A}P(C)P(A|C)P(D|A)\\
    &=P(C)\sum_{A}P(A|C)P(D|A)
\end{align*}
In order for this to be independent we must have that $\sum_{A}P(A|C)P(D|A)=P(D)$
This will occur if and only if $P(D|A)=P(D)$ or $P(A|C)=P(A)$, this generally will not hold true because that would
mean that either $A$ is independent of $C$ or $A$ is independent of $D$, neither of which should hold true.
Therefore $C$ and $D$ are not independent.
\subsection*{(b)}
 





\end{document}