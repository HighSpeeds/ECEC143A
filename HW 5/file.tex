\UseRawInputEncoding
\documentclass[12pt]{article}
\title{ECE C143A Homework 3}
\usepackage{subcaption}
\author{Lawrence Liu}
\usepackage{graphicx}
\usepackage{amsmath}
\usepackage{bm}
\usepackage{pdfpages}
\newcommand{\Laplace}{\mathscr{L}}
\setlength{\parskip}{\baselineskip}%
\setlength{\parindent}{0pt}%
\usepackage{xcolor}
\usepackage{listings}
\definecolor{backcolour}{rgb}{0.95,0.95,0.92}
\usepackage{amssymb}
\lstdefinestyle{mystyle}{
    backgroundcolor=\color{backcolour}}
\lstset{style=mystyle}

\begin{document}
\maketitle
\section*{Problem 1}
\subsection*{(a)}
$P(N=0)=1-0.25=\boxed{0.75}$
\subsection*{(b)}
We want $P(N=0|R=0)$, we know that
$$P(R=1|N=0)=P(E=0)P(R=1|E=0,N=0)+P(E=1)P(R=1|E=1,N=0)=0.01$$
$P(R=0|N=0)=0.99$ and thus from bayes law we have 
\begin{align*}
    P(N=0|R=0)&=P(R=0|N=0)\frac{P(N=0)}{P(R=0)}\\
    &=0.9\frac{0.1\cdot 0.25}{P(R=0)}
\end{align*}
To find $P(R=0)$ we must find $P(R=1)$,
\begin{multline*}
    P(R=1)=P(E=1)P(N=1)P(R=1|E=1,N=1)\\+P(E=1)P(N=0)P(R=1|E=1,N=0)
    \\+P(E=0)P(N=1)P(R=1|E=0,N=1)\\+P(E=0)P(N=0)P(R=1|E=0,N=0)
\end{multline*}
\begin{multline*}
    P(R=1)=0.9\cdot0.25\cdot 1+0.1\cdot0.25\cdot0.1+0.1\cdot 0.75\cdot 0.1=0.235
\end{multline*}
Therefore $P(R=0)=1-P(R=1)=0.765$, thus
$$P(N=0|R=0)=0.99\frac{0.25}{P(R=0)}=\boxed{0.3235}$$
\subsection*{(c)}
We have
\begin{align*}
    P(E=0,N=0|R=0)&=\frac{P(E=0,N=0,R=0)} {P(R=0)}\\
    &=\frac{P(R=0|E=0,N=0)P(E=0)P(N=0)} {P(R=0)}\\
    &=\frac{(1-P(R=1|E=0,N=0))P(E=0)P(N=0)} {P(R=0)}\\
    &=\frac{0.9\cdot0.1\cdot0.75}{0.765}\\
    &=\boxed{0.088}
\end{align*}
This intuitively makes sense because we need two conidtions to occur, equipment broken and neuron spike not occuring.


\end{document}